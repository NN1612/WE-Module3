\documentclass{article}
\usepackage{geometry}
\usepackage{lipsum}
\usepackage{hyperref}

\geometry{
    a4paper,
    total={170mm,257mm},
    left=20mm,
    top=20mm,
}

\begin{document}

\begin{titlepage}
    \centering
    \vspace*{2cm}
    {\scshape\LARGE WE MODULE 3 \par}
    \vspace{1cm}
    {\scshape\Large Technical Report\par}
    \vspace{1.5cm}
    {\huge\bfseries Developing Strategies for the Bidding Card Game "Diamonds" with GenAI\par}
    \vspace{0.5cm}
    {\Large\itshape NAVYA NAYER\par}
    \vfill
    {\large \today\par}
\end{titlepage}

\section{Abstract}
The bidding card game "Diamonds" presents an intriguing challenge for artificial intelligence (AI) systems, blending strategic bidding and card management. In this report, I delve into the process of developing strategies for playing "Diamonds" with GenAI, a computer program designed to learn and adapt through iterative gameplay.

\section{Introduction}
The bidding card game "Diamonds" presents an intriguing challenge for artificial intelligence (AI) systems, blending strategic bidding and card management. In this report, I delve into the process of developing strategies for playing "Diamonds" with GenAI, a computer program designed to learn and adapt through iterative gameplay.

\section{Problem Statement}
The objective is to equip GenAI with the skills to play "Diamonds" effectively, enhancing its chances of winning against human opponents. This involves teaching GenAI the rules of the game, guiding its strategic development, and refining its bidding and card management strategies.

\section{Teaching GenAI the Game}
The journey commenced with the initial prompt, which outlined the foundational rules of the "Diamonds" game. This prompt provided GenAI with a comprehensive understanding of bidding mechanics, winning diamond cards, and scoring points. Each player receives a suit of cards other than diamonds, with diamond cards shuffled and auctioned one by one. Players bid with one of their cards face down, with the highest bid winning the diamond card and receiving its points. In case of ties, points are divided equally among the highest bidders.

\vspace{0.5cm} % Adjust the space between paragraphs as needed

Additionally, teaching GenAI involved a demo game played between me and GenAI, where I pointed out errors it made during gameplay. One such challenge GenAI encountered was maintaining the list of remaining cards after bidding for a card. Accurate tracking of remaining cards is crucial for informed bidding decisions and strategic card management.

\vspace{0.5cm} % Adjust the space between paragraphs as needed

Moreover, GenAI faced difficulty in writing the code to implement bidding strategies and card management techniques. Understanding the intricacies of translating strategic concepts into executable code proved challenging initially. However, with continued guidance and practice, GenAI made significant strides in overcoming this hurdle, demonstrating resilience and adaptability in learning the necessary programming skills. While there is still progress to be made to optimize the code fully.

\vspace{0.5cm} % Adjust the space between paragraphs as needed

For further details, see the GenAI transcript \href{https://docs.google.com/document/d/1eL58ZQdyULYBA-CMjfxMR6xRTC0HlldTMRjW1cI0_RM/edit?usp=sharing}{here}.


\section{Iterating on Strategy}
Following the elucidation of the rules, GenAI embarked on iterative strategy development. Guided prompts and interactive gameplay sessions allowed GenAI to refine its bidding and card management tactics over time. Through analysis of gameplay data and adjustments based on performance, GenAI evolved its strategies, demonstrating adaptability and strategic prowess.

\section{Analysis}
GenAI's progress was hindered by the difficulty in writing code to implement bidding strategies and card management techniques. The complexity of translating strategic concepts into executable code posed a significant challenge. However, this obstacle presents an opportunity for improvement. Future iterations of GenAI can benefit from enhanced coding capabilities, achieved through advanced training in programming languages and algorithms. Additionally, collaboration with AI experts and software engineers can provide valuable insights and guidance in overcoming coding challenges.

\section{Future Directions}
Moving forward, further refinement of GenAI's bidding and card management strategies will be essential. Additionally, exploring advanced AI techniques such as reinforcement learning could enhance GenAI's ability to adapt to changing game conditions and opponents' strategies. Continued collaboration and experimentation will drive GenAI's evolution as a proficient player in the "Diamonds" game.

\section{Conclusion}
The journey of developing strategies for playing "Diamonds" with GenAI is a testament to the power of guided learning and iterative refinement. Beginning with the foundational rules explanation and a demo game, GenAI evolved into a formidable opponent, capable of competing strategically against human adversaries. As AI continues to advance, the strategies developed for games like "Diamonds" serve as a testament to the potential of AI-driven approaches in mastering complex strategic challenges.

\end{document}
